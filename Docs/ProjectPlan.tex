\documentclass[a4paper, 11pt]{article}
\usepackage{color}
\usepackage{fancyhdr}
\usepackage{float}
\usepackage{stfloats}
\usepackage{placeins}
\usepackage{tabularray}
\usepackage{xcolor,colortbl}
\usepackage[top=2.5cm, bottom=2cm, left = 2.5cm, right = 2.5cm]{geometry} 
\geometry{a4paper} 
\usepackage[utf8]{inputenc}
\usepackage{textcomp}
\usepackage{graphicx} 
\usepackage{amsmath,amssymb}  
\usepackage{bm}  
\usepackage[pdftex,bookmarks,colorlinks,breaklinks]{hyperref} 
\hypersetup{linkcolor=MSBlue,citecolor=black,filecolor=black,urlcolor=black} % black links, for printed output
\usepackage{memhfixc} 
\usepackage{pdfsync}  
\usepackage{xcolor}
\usepackage{titlesec}
\usepackage{tocloft}
\usepackage{rotating}
\usepackage{array}
\usepackage{adjustbox}
\usepackage{hhline}

\definecolor{MSBlue}{RGB}{47, 84, 150}
\definecolor{MSGray}{RGB}{128, 128, 128}

\renewcommand{\cftsecfont}{\fontfamily{qag}\selectfont\bfseries} 
\renewcommand{\cftsecpagefont}{\fontfamily{qag}\selectfont\bfseries\color{MSBlue}} 
\renewcommand{\cfttoctitlefont}{\fontfamily{qag}\selectfont\LARGE\bfseries}               
\renewcommand{\familydefault}{phv}


\fancypagestyle{titlepage}{
  \fancyhf{}
  \rfoot{\fontfamily{qag}\fontsize{11pt}{0pt}\selectfont\color{MSGray} }
  \renewcommand{\headrulewidth}{0pt}
  \renewcommand\footrulewidth{0pt}
}



\pagestyle{fancy}
\renewcommand{\headrulewidth}{0pt}
\renewcommand{\footrulewidth}{0pt}
\setlength{\headheight}{15pt}
\rhead{\fontfamily{qag}\fontsize{10pt}{12pt}\selectfont\color{MSGray} 04.06.23}
\lhead[]{}
\fancyfoot[C]{\fontsize{10pt}{10pt}\selectfont\thepage} 



\titleformat{\section}
  {\fontfamily{qag}\selectfont\LARGE\bfseries\color{MSBlue}}
  {\thesection}{0.5em}{}
  
  
\titleformat{\subsection}
  {\fontfamily{qag}\selectfont\Large\mdseries\color{MSBlue}}
  {\thesubsection}{0.5em}{}

\titleformat{\subsubsection}
  {\fontfamily{qag}\selectfont\large\mdseries\color{MSBlue}}
  {\thesubsubsection}{0.5em}{}

\titlespacing\subsubsection{0pt}{12pt plus 4pt minus 2pt}{0pt plus 2pt minus 2pt}

\linespread{1.2} 

\begin{document}

\begin{titlepage}
  \thispagestyle{titlepage}
  \begin{center} 
    \end{center}


	\setlength{\parindent}{0pt}
	\vspace*{.15\textheight}
	\medbreak
	{\fontfamily{qag}\Huge\bfseries\color{MSBlue}Project Plan\par}
    {\fontfamily{qag}\LARGE\bfseries<<Project Name>>\par}
	\bigbreak
    {\fontfamily{qag}\LARGE<<Client>>\par}
    {\fontfamily{qag}\large<<City>>\par}


    \bigbreak
	{Michał Raczkowski\par}
    \smallbreak
    {\small  \par}
    \smallbreak
    {\small 4465024\par}

    \vfill
\begin{table}[b]
  \centering
  \begin{tblr}{
    width = \linewidth,
    colspec = {Q[60]Q[120]},
    hlines,
    vlines,
  }
   Date  & 04-03-2024 \\ 
   Version  & 0.1 \\        
   Status  & In Progress \\           
   Author  & Michal Raczkowski\\           
  \end{tblr}
\end{table}
\end{titlepage}



\pagebreak

{\Large\noindent Versions}

\begin{table}[h]
    \centering
    \begin{tblr}{
      width = \linewidth,
      colspec = {Q[70]Q[80]Q[120]Q[248]Q[100]},
      hlines,
      vlines,
    }
    \textbf{Version} & \textbf{Date} & \textbf{Author} & \textbf{Amendments} & \textbf{Status} \\
             0.1     &      04-03-2024       &   Michał Raczkowski  &  - & In Progress \\ 
                   &            & &  \\
  
    \end{tblr}
  \end{table}

  {\Large\noindent Communication}


  \begin{table}[h]
    \centering
    \begin{tblr}{
      width = \linewidth,
      colspec = {Q[60]Q[70]Q[327]},
      hlines,
      vlines,
    }
    \textbf{Version} & \textbf{Date} & \textbf{To} \\
        0.1          &       04-03-2024  &     \\
                    &             & \\
  
    \end{tblr}
  \end{table}

\pagebreak

\tableofcontents




\pagebreak


\section{Project Assignment}


\subsection{Context}
Alltrons is a premier research and development partner, specializing in transforming ideas into scalable products. With a focus on IoT and robotics, Alltrons leverages a diverse team of specialists across UI/UX design, software development (including frontend, backend, and embedded systems), electrical engineering, mechanics, and business strategy. This multidisciplinary approach eliminates the need for companies to engage multiple parties or expand internal teams for R\&D projects. Alltrons embodies innovation, providing comprehensive solutions from concept development to production, streamlining the path from ideation to market-ready products.

\subsection{Goal of the project}
The goal of this project is to evolve our current IoT solution framework from being based on general-purpose, Linux-based devices (like Raspberry Pi) to a more specialized, embedded architecture (using chips such as Quectel, SIMCom, or nRF52840). This evolution includes creating a desktop application dedicated to programming the new embedded hardware, enabling Over-The-Air (OTA) updates for seamless, future-proof upgrades.
\smallbreak
The motivation behind this transition is driven by the necessity for a solution that is not only more cost-efficient but also superior in terms of energy and resource efficiency. The move away from a Linux-based framework allows us to address these needs while expanding our potential client base. Embedded systems offer a targeted approach to IoT development, aligning with the market's demand for leaner, more efficient technologies.
\smallbreak
The desktop application will play a pivotal role in this project, offering users the ability to configure embedded devices through a USB interface for initial setups such as GSM provider preferences, server connections, and network configurations. It's crucial to note that while the application facilitates initial device programming and configuration, the OTA update mechanism operates independently of this utility.
\smallbreak
This project's overarching aim is to refine our IoT solutions' efficiency and market applicability, positioning us to cater to a broader array of client needs with a more refined, cost-effective, and technologically advanced development framework.

\subsection{The assignment}
The assignment involves creating a desktop application designed for the configuration and adjustment of embedded devices within our existing infrastructure. This application will serve as a pivotal tool for deploying a wide range of IoT solutions. The primary function of this desktop application is to provide a user-friendly interface for configuring embedded hardware, tailored to the specific needs of various IoT applications.

\subsection{Scope}


Scope of the project:
  \definecolor{Silver}{rgb}{0.752,0.752,0.752}
  \begin{table}[h]
  \centering
  \begin{tblr}{
    width = \linewidth,
    colspec = {Q[334]Q[280]},
    cell{1}{2} = {Silver},
    hlines,
    vlines,
  }
  Includes                                                                            & Excludes                                                                \\
  Design and development of the desktop application for embedded device configuration & Development of the OTA update mechanism (handled separately)            \\
  Integration with existing embedded hardware platforms                               & Creation of new embedded devices or hardware modifications              \\
  USB interface development for device configuration                                  & Long-term maintenance and support beyond initial deployment             \\
  User interface design for easy navigation and operation                             & evelopment of additional applications for other platforms (mobile, web) \\
  Testing and deployment of the application                                           & Expansion to non-IoT embedded systems or unrelated hardware             \\
  Documentation                                                                       &                                                                         
  \end{tblr}
  \end{table}


\subsection{Conditions}
The preferred frontend framework should be \textbf{React} because it is the main frontend framework used within the company.
\subsection{Finished products}
A Product Breakdown Structure of the end and intermediate products that the project will deliver with a short description in text of each product.The end products are more than the project plan and the product itself.Also, for example, requirements and architecture documents and research and test reports are typical parts of a PBS.These documents are important for the relevant stakeholders during development as well as during the transfer and during the management phase.During the project you can change the PBS and you can add or remove products in consultation
\pagebreak
\subsection{Research questions}
To ensure the successful development and deployment of the desktop application for configuring and adjusting existing infrastructure embedded devices for IoT solutions, the following research questions have been identified as critical:
\begin{enumerate}
  \item \textbf{Compatibility and Integration}:
  \begin{itemize}
    \item What are the primary technical challenges in achieving seamless compatibility between the desktop application and a wide range of embedded devices (e.g., Quectel, SIMCom, nRF52840)?
    \item How can the application support diverse IoT solutions and configurations
\end{itemize}
  \item \textbf{Interface Design}:
  \begin{itemize}
    \item How can the application provide a streamlined process for configuring critical parameters (GSM provider, server, network settings) via USB interface?
    \item How to design a user interface that simplifies the configuration process for users with varying levels of technical expertise?
\end{itemize}
  \item \textbf{Security}:
  \begin{itemize}
    \item What security measures are necessary to protect the configuration process and sensitive device settings from unauthorized access?
    \item How can the application support diverse IoT solutions and configurations
\end{itemize}
  \item \textbf{Technical Feasibility and Development Strategy}:
  \begin{itemize}
    \item What development frameworks and technologies are most suited for building a cross-platform desktop application that meets the project's requirements?
    \item How can the project leverage existing infrastructure and tools to accelerate development and reduce costs?
\end{itemize}
\end{enumerate}


\section{Approach and Plannings}
\subsection{Approach}
Indicate here which method you follow in your project plan, for example whether you use a waterfall or scrum method.also indicate how you will approach the problem definition phase and completion phase. With a scrum approach you can think of length of sprints, set-up of your sprints, stand-up, set-up of demos, retrospective, etc.
\subsubsection{Test approach}
If applicable: how is testing designed. What is the approach? And why? Also include any approach to (Code) reviews in this
\subsection{Research methods}
Describe (per research question and for the entire project) which methods (see ictresearchmethods.nl or cmdmethods.nl) you will use to answer the most important questions within your project (= how you will substantiate the most important choices).Do not only mention the method, but also briefly explain how you will use it (e.g. who will you interview and for what purpose?).Of course, your approach can still be adjusted during your internship
\subsection{Learning outcomes}
Discuss how you are going to demonstrate each learning outcome in the project. The easiest way is to think about which of the professional products you are going to use as evidence for each of the learning outcomes.
\subsection{Breakdown of the project}
Show the rough breakdown in phases or sprints of the project here
\subsection{Time plan}
Depending on your project method, you will be able to work out the phasing in more or less detail. Below is a possible table that you can use for this.
Note that with an agile approach, most projects still have a problem analysis/orientation phase (or 'sprint 0'), as well as a completion/evaluation phase.
Also make sure that you reserve enough time for your portfolio and start on time.
\begin{table}[h]
    \centering
    \begin{tblr}{
      width = \linewidth,
      colspec = {Q[45]Q[20]Q[15]Q[15]},
      hlines,
      vlines,
    }
    \textbf{Phasing} & \textbf{Effort} & \textbf{Start} & \textbf{Ready}  \\
              1.    &               \\
               2.     &              \\
               3.     &              \\
    \end{tblr}
  \end{table}
\section{Project Organization}
\subsection{Team members}
Describe the organization of the project with its immediate environment. An organization chart can be displayed for clarification. Indicate in descriptive form which roles are included in the organization chart with the associated authorities and responsibilities. It must be clear who is authorized to do what and what can be expected of whom. Indicate who is involved in your project and what his/her function is and what the role is within your project. For example, someone with the function 'manager of department X' can have the role of Product Owner in your project. In this project, both the internship/graduation organization and Fontys are stakeholders. So include your internship teachers and yourself in this schedule.
\definecolor{GrayNickel}{rgb}{0.752,0.749,0.737}
\begin{table}[H]
\centering
\begin{tblr}{
  width = \linewidth,
  colspec = {Q[320]Q[230]Q[280]},
  row{1} = {GrayNickel},
  hlines,
  vlines,
}
Name + Phone + e-mail                                                                & Role/tasks                  & Availability                             \\
{Kayle Knops\\kayle.knops@alltrons.com}                                              & CEO/Control                 & 1 hour/week                              \\
{Luis Pellicer Collado\\\textit{}luis.pellicer@alltrons.com}                         & Tech Lead Embedded Software & About 3 hours per week,~depends on needs \\
{Kamyar Khodayari\\kamyar.khodayari@alltrons.com}                                    & System Solution Architect   & 1 hour/week~depends on needs             \\
{Schürgers,Frank F.P.\\f.schurgers@fontys.nl}                                        & University coach            &                                          \\
{Michal Raczkowski\\m.raczkowski@student.fontys.nl\\michael.raczkowski@alltrons.com} & Intern                      & 40h/week                                 
\end{tblr}
\end{table}

  \subsection{Communication}
  Indicate which communication/attunements there are. Think of coordination with company supervisor, teacher supervisor and other stakeholders. How and how often do these attunements take place?
  \subsection{Test environment}
  omit this section if not applicable
  \smallbreak
  Describe what the test environment looks like. A picture generally gives the best overview. Also record to what extent you use a CI/CD environment (self-developed or using an existing system)
  \smallbreak
  Describe which products are included in the test environment. These can be products that the project produces, but also external products that are necessary to perform the test approach (e.g. computers)
  \subsection{Configuration management}
  omit this section if not applicable

  \smallbreak
  Describe how the archive is set up (for example your GIT repository structure with branching strategy). Pictures about, for example, your branching strategy can help with this. If possible, describe which baselines and releases you foresee



\section{Finance and Risks}
\subsection{Cost budgets}
If specific costs have to be incurred for your project, please indicate these. Think of extra hardware or software investments. Regular matters such as workplace, your internship allowance, etc. do not have to be included.
\subsection{Risks and fall-back activities}
Define risks. What have you already included in the plan to limit or prevent the risk? What choice is made if the risk does unexpectedly occur? Think of organizational risks (such as the sudden leave of the company supervisor) as well as more substantive risks (for example, what to do if you find out during your internship that it is better for the company to purchase an external application instead of the application to be developed)
\smallbreak
Think of real risks that can actually influence your project. For example, there may be a risk that your company supervisor will be absent, for example due to illness or because he is going to do something else. Is there a backup in the company
\begin{table}[H]
    \centering
    \begin{tblr}{
      width = \linewidth,
      colspec = {Q[50]Q[50]Q[50]},
      hlines,
      vlines,
    }
    \textbf{Risk} & \textbf{Prevention activities included in plan} & \textbf{Fall-back Activities}  \\
      1 &       &        \\
      2 &       &        \\
      3 &       &        \\

    \end{tblr}
  \end{table}
  \section{Other}
  Describe here everything that is relevant but that you cannot put elsewhere in the document


\end{document}