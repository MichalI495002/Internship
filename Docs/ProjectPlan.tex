\documentclass[a4paper, 11pt]{article}
\usepackage{color}
\usepackage{fancyhdr}
\usepackage{float}
\usepackage{stfloats}
\usepackage{placeins}
\usepackage{tabularray}
\usepackage{xcolor,colortbl}
\usepackage[top=2.5cm, bottom=2cm, left = 2.5cm, right = 2.5cm]{geometry} 
\geometry{a4paper} 
\usepackage[utf8]{inputenc}
\usepackage{textcomp}
\usepackage{graphicx} 
\usepackage{amsmath,amssymb}  
\usepackage{bm}  
\usepackage[pdftex,bookmarks,colorlinks,breaklinks]{hyperref} 
\hypersetup{linkcolor=MSBlue,citecolor=black,filecolor=black,urlcolor=black} % black links, for printed output
\usepackage{memhfixc} 
\usepackage{pdfsync}  
\usepackage{xcolor}
\usepackage{titlesec}
\usepackage{tocloft}
\usepackage{rotating}
\usepackage{array}
\usepackage{adjustbox}
\usepackage{hhline}

\definecolor{MSBlue}{RGB}{47, 84, 150}
\definecolor{MSGray}{RGB}{128, 128, 128}

\renewcommand{\cftsecfont}{\fontfamily{qag}\selectfont\bfseries} 
\renewcommand{\cftsecpagefont}{\fontfamily{qag}\selectfont\bfseries\color{MSBlue}} 
\renewcommand{\cfttoctitlefont}{\fontfamily{qag}\selectfont\LARGE\bfseries}               
\renewcommand{\familydefault}{phv}


\fancypagestyle{titlepage}{
  \fancyhf{}
  \rfoot{\fontfamily{qag}\fontsize{11pt}{0pt}\selectfont\color{MSGray} }
  \renewcommand{\headrulewidth}{0pt}
  \renewcommand\footrulewidth{0pt}
}



\pagestyle{fancy}
\renewcommand{\headrulewidth}{0pt}
\renewcommand{\footrulewidth}{0pt}
\setlength{\headheight}{15pt}
\rhead{\fontfamily{qag}\fontsize{10pt}{12pt}\selectfont\color{MSGray} 04.06.23}
\lhead[]{}
\fancyfoot[C]{\fontsize{10pt}{10pt}\selectfont\thepage} 



\titleformat{\section}
  {\fontfamily{qag}\selectfont\LARGE\bfseries\color{MSBlue}}
  {\thesection}{0.5em}{}
  
  
\titleformat{\subsection}
  {\fontfamily{qag}\selectfont\Large\mdseries\color{MSBlue}}
  {\thesubsection}{0.5em}{}

\titleformat{\subsubsection}
  {\fontfamily{qag}\selectfont\large\mdseries\color{MSBlue}}
  {\thesubsubsection}{0.5em}{}

\titlespacing\subsubsection{0pt}{12pt plus 4pt minus 2pt}{0pt plus 2pt minus 2pt}

\linespread{1.2} 

\begin{document}

\begin{titlepage}
  \thispagestyle{titlepage}
  \begin{center} 
    \end{center}


	\setlength{\parindent}{0pt}
	\vspace*{.15\textheight}
	\medbreak
	{\fontfamily{qag}\Huge\bfseries\color{MSBlue}Project Plan\par}
    {\fontfamily{qag}\LARGE\bfseries<<Project Name>>\par}
	\bigbreak
    {\fontfamily{qag}\LARGE<<Client>>\par}
    {\fontfamily{qag}\large<<City>>\par}


    \bigbreak
	{Michał Raczkowski\par}
    \smallbreak
    {\small  \par}
    \smallbreak
    {\small 4465024\par}

    \vfill
\begin{table}[b]
  \centering
  \begin{tblr}{
    width = \linewidth,
    colspec = {Q[60]Q[120]},
    hlines,
    vlines,
  }
   Date & Date \\ 
   Version  & Version \\        
   Status  & Status \\           
   Author  & Author \\           
  \end{tblr}
\end{table}
\end{titlepage}



\pagebreak

{\Large\noindent Versions}

\begin{table}[h]
    \centering
    \begin{tblr}{
      width = \linewidth,
      colspec = {Q[70]Q[80]Q[120]Q[248]Q[100]},
      hlines,
      vlines,
    }
    \textbf{Version} & \textbf{Date} & \textbf{Author} & \textbf{Amendments} & \textbf{Status} \\
                  &             &    &   \\
                   &            & &  \\
  
    \end{tblr}
  \end{table}

  {\Large\noindent Communication}


  \begin{table}[h]
    \centering
    \begin{tblr}{
      width = \linewidth,
      colspec = {Q[60]Q[70]Q[327]},
      hlines,
      vlines,
    }
    \textbf{Version} & \textbf{Date} & \textbf{To} \\
                  &           &     \\
                    &             & \\
  
    \end{tblr}
  \end{table}

\pagebreak

\tableofcontents




\pagebreak


\section{Project Assignment}


\subsection{Context}
Briefly describe the company and the context of the assignment.Provide information about the products and services of the company that your assignment focuses on. If you work for an external client of your client  think of a client of an internship company describe them in the same way. In addition, indicate the concrete reason for the assignment and what developments are taking place in the company or the market that lead to the assignment
\subsection{Goal of the project}
Describe the goal of the project here. Think of:
What is the problem that needs to be solved or what is the opportunity that needs to be used?
What does the desired situation look like?
What benefits does the project offer?
What possibilities (capabilities, facilities) does the product or project result offer?
\smallbreak
NB To make your goals as concrete as possible, you should already have a good idea of the problem. What exactly is the issue? What is the problem to be solved or what is the challenge? Why is this question there? What is the urgency? What caused it? What are the consequences if nothing is done? And what has already been done to arrive at an answer? It is essential that you look critically at the client's needs. Is the problem outlined actually the problem? And is your client's question actually the right solution? Ask critical questions and try to arrive at the correct problem statement together with the client. If more research is needed to determine this, include this in your approach.
\subsection{The assignment}
Formulate the assignment. The assignment definition itself should consist of a text that is as short aand concise as possible in which the assignment is clearly formulated. What are the specific requirements/wishes of the client at the start of the project? What are the minimum (quality) requirements that the end result must meet? You may already provide a list of functional and non-functional requirements for the end product as an attachment>
\subsection{Scope}
Indicate the scope.If necessary, make a context diagram for clarification that shows the relationships with other systems and the environment.This section should also describe what will not be delivered.For example, if you agree to deliver a high fidelity prototype, then (part of) the implementation and management falls outside your scope.
Make this as concrete as possible so that there are no misunderstandings between you and your client.
\definecolor{Silver}{rgb}{0.752,0.752,0.752}
\begin{table}[h]
    \centering
    \begin{tblr}{
      width = \linewidth,
      colspec = {Q[45]Q[50]},
      hlines,
      vlines,
      cell{1}{2} = {Silver},
    }
    \textbf{The project includes:} & \textbf{The project does not include:}  \\
              1.    &               \\
               2.     &              \\
  
    \end{tblr}
  \end{table}
\subsection{Conditions}
Indicate, where necessary, what the preconditions are.For example, consider technology set by the company.Note that a critical attitude remains important here!>
\subsection{Finished products}
A Product Breakdown Structure of the end and intermediate products that the project will deliver with a short description in text of each product.The end products are more than the project plan and the product itself.Also, for example, requirements and architecture documents and research and test reports are typical parts of a PBS.These documents are important for the relevant stakeholders during development as well as during the transfer and during the management phase.During the project you can change the PBS and you can add or remove products in consultation
\subsection{Research questions}
Describe the most important research questions you want to answer during your internship. Define a main question with sub-questions derived from it.Keep in mind that you will be doing investigative work during your entire internship, and that your questions will therefore concern your entire trajectory.During your internship/graduation, more research questions may be of interest and others may turn out to be less relevant.Describe only the key research questions that will have the greatest impact on your project.Other research aspects can be elaborated in more detail during your internship and can then be explained with a short substantiation in your portfolio or orally
\section{Approach and Plannings}
\subsection{Approach}
Indicate here which method you follow in your project plan, for example whether you use a waterfall or scrum method.also indicate how you will approach the problem definition phase and completion phase. With a scrum approach you can think of length of sprints, set-up of your sprints, stand-up, set-up of demos, retrospective, etc.
\subsubsection{Test approach}
If applicable: how is testing designed. What is the approach? And why? Also include any approach to (Code) reviews in this
\subsection{Research methods}
Describe (per research question and for the entire project) which methods (see ictresearchmethods.nl or cmdmethods.nl) you will use to answer the most important questions within your project (= how you will substantiate the most important choices).Do not only mention the method, but also briefly explain how you will use it (e.g. who will you interview and for what purpose?).Of course, your approach can still be adjusted during your internship
\subsection{Learning outcomes}
Discuss how you are going to demonstrate each learning outcome in the project. The easiest way is to think about which of the professional products you are going to use as evidence for each of the learning outcomes.
\subsection{Breakdown of the project}
Show the rough breakdown in phases or sprints of the project here
\subsection{Time plan}
Depending on your project method, you will be able to work out the phasing in more or less detail. Below is a possible table that you can use for this.
Note that with an agile approach, most projects still have a problem analysis/orientation phase (or 'sprint 0'), as well as a completion/evaluation phase.
Also make sure that you reserve enough time for your portfolio and start on time.
\begin{table}[h]
    \centering
    \begin{tblr}{
      width = \linewidth,
      colspec = {Q[45]Q[20]Q[15]Q[15]},
      hlines,
      vlines,
    }
    \textbf{Phasing} & \textbf{Effort} & \textbf{Start} & \textbf{Ready}  \\
              1.    &               \\
               2.     &              \\
               3.     &              \\
    \end{tblr}
  \end{table}
\section{Project Organization}
\subsection{Team members}
Describe the organization of the project with its immediate environment. An organization chart can be displayed for clarification. Indicate in descriptive form which roles are included in the organization chart with the associated authorities and responsibilities. It must be clear who is authorized to do what and what can be expected of whom. Indicate who is involved in your project and what his/her function is and what the role is within your project. For example, someone with the function 'manager of department X' can have the role of Product Owner in your project. In this project, both the internship/graduation organization and Fontys are stakeholders. So include your internship teachers and yourself in this schedule.
\begin{table}[H]
    \centering
    \begin{tblr}{
      width = \linewidth,
      colspec = {Q[50]Q[20]Q[50]Q[50]},
      hlines,
      vlines,
    }
    \textbf{Name + Phone + e-mail} & \textbf{Abbr.} & \textbf{Role/tasks} & \textbf{Availability}  \\
    Contact details   &      Sometimes it is useful to indicate a role or person with an abbreviation & Mention the role or any specific tasks & What availability of the person is necessary (e.g. 3 days a week, during phase 2)         \\

    \end{tblr}
  \end{table}

  \subsection{Communication}
  Indicate which communication/attunements there are. Think of coordination with company supervisor, teacher supervisor and other stakeholders. How and how often do these attunements take place?
  \subsection{Test environment}
  omit this section if not applicable
  \smallbreak
  Describe what the test environment looks like. A picture generally gives the best overview. Also record to what extent you use a CI/CD environment (self-developed or using an existing system)
  \smallbreak
  Describe which products are included in the test environment. These can be products that the project produces, but also external products that are necessary to perform the test approach (e.g. computers)
  \subsection{Configuration management}
  omit this section if not applicable

  \smallbreak
  Describe how the archive is set up (for example your GIT repository structure with branching strategy). Pictures about, for example, your branching strategy can help with this. If possible, describe which baselines and releases you foresee



\section{Finance and Risks}
\subsection{Cost budgets}
If specific costs have to be incurred for your project, please indicate these. Think of extra hardware or software investments. Regular matters such as workplace, your internship allowance, etc. do not have to be included.
\subsection{Risks and fall-back activities}
Define risks. What have you already included in the plan to limit or prevent the risk? What choice is made if the risk does unexpectedly occur? Think of organizational risks (such as the sudden leave of the company supervisor) as well as more substantive risks (for example, what to do if you find out during your internship that it is better for the company to purchase an external application instead of the application to be developed)
\smallbreak
Think of real risks that can actually influence your project. For example, there may be a risk that your company supervisor will be absent, for example due to illness or because he is going to do something else. Is there a backup in the company
\begin{table}[H]
    \centering
    \begin{tblr}{
      width = \linewidth,
      colspec = {Q[50]Q[50]Q[50]},
      hlines,
      vlines,
    }
    \textbf{Risk} & \textbf{Prevention activities included in plan} & \textbf{Fall-back Activities}  \\
      1 &       &        \\
      2 &       &        \\
      3 &       &        \\

    \end{tblr}
  \end{table}
  \section{Other}
  Describe here everything that is relevant but that you cannot put elsewhere in the document


\end{document}